\documentclass[11pt,a4paper]{article}
\usepackage{geometry}
\usepackage{enumitem}
\usepackage{fancyhdr}
\geometry{margin=1in}

\pagestyle{fancy}
\fancyhf{}
\rhead{Name: P. Muskan \\ ID: COMETFWC035}
\lhead{IIITB COMET}
\renewcommand{\headrulewidth}{0.4pt}

\title{\textbf{Weekly Learning Report\\5G NR Layer 2 and Layer 3 Protocols}}
\author{Internship Technical Summary}
\date{\today}

\begin{document}
\maketitle

\section*{Overview}
This report summarizes the key technical concepts learned during the past week of the internship. The focus was on understanding the functionality and interaction of 5G NR Layer 2 and Layer 3 protocols, particularly SDAP, PDCP, RLC, MAC, Advanced MAC procedures, and RRC. The learning emphasized practical reasoning behind protocol design choices rather than specification-level depth.

\section{SDAP Layer}
This week, I learned how SDAP enables QoS handling between the 5G Core and the radio access network.

\begin{itemize}[itemsep=2pt]
    \item Understood QoS Flow to DRB mapping using QFI.
    \item Learned the purpose of Reflective QoS in simplifying uplink QoS handling.
    \item Observed how SDAP header usage can be optimized to reduce overhead.
\end{itemize}

\section{PDCP Layer}
The PDCP layer learning focused on security, packet ordering, and mobility support.

\begin{itemize}[itemsep=2pt]
    \item Learned how ciphering and integrity protection are applied using PDCP COUNT.
    \item Understood PDCP sequence numbering and reordering mechanisms.
    \item Studied duplicate detection and data forwarding during handover.
    \item Gained a basic understanding of ROHC and its benefit for VoIP traffic.
\end{itemize}

\section{RLC Layer}
A major part of the learning involved understanding why multiple RLC modes exist.

\begin{itemize}[itemsep=2pt]
    \item Learned differences between RLC TM, UM, and AM.
    \item Understood segmentation, reassembly, and sequence numbering.
    \item Studied ARQ operation in RLC AM using polling and status reports.
    \item Learned the role of key timers such as t-Reassembly and t-PollRetransmit.
\end{itemize}

\section{MAC Layer}
The MAC layer learning focused on scheduling and uplink resource management.

\begin{itemize}[itemsep=2pt]
    \item Understood the role of the MAC scheduler in balancing throughput and QoS.
    \item Learned Logical Channel Prioritization using priority and PBR.
    \item Studied Buffer Status Reports (BSR) and their triggering conditions.
    \item Learned how Power Headroom Reports (PHR) influence uplink scheduling.
\end{itemize}

\section{Advanced MAC Procedures}
Advanced MAC concepts were studied mainly from a power efficiency and latency perspective.

\begin{itemize}[itemsep=2pt]
    \item Learned DRX operation and its impact on UE power saving and latency.
    \item Understood the purpose of Semi-Persistent Scheduling for periodic traffic.
    \item Studied HARQ retransmissions and basic carrier aggregation concepts.
\end{itemize}

\section{RRC Layer}
The RRC layer learning focused on connection management and mobility.

\begin{itemize}[itemsep=2pt]
    \item Learned RRC states: IDLE, CONNECTED, and INACTIVE.
    \item Understood how RRC\_INACTIVE reduces signaling overhead.
    \item Studied basic measurement reporting and handover triggering.
\end{itemize}

\section*{Conclusion}
During this week, I gained a concise and structured understanding of how 5G NR Layer 2 and Layer 3 protocols work together to provide QoS, security, reliability, power efficiency, and mobility. This learning helped me connect protocol concepts with practical system behavior, forming a strong foundation for further work in 5G protocol implementation and analysis.

\end{document}
