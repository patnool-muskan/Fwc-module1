\documentclass{article}
\usepackage{amsmath}
\usepackage{amsfonts}
\usepackage{amssymb}
\usepackage{fancyhdr}
\usepackage{graphicx}
\begin{document}
\pagestyle{fancy}
\fancyhf{}
\fancyhead[L]{\includegraphics[height=1cm]{IIITB-COMET-Logo.png}}
\fancyhead[R]{Name:P.Muskan\\
ID: cometfwc035}

\begin{center}
\textbf{OLYMPIAD 1974-76}
\end{center}
\section*{Sixteenth International Olympiad, 1974}
\begin{enumerate}
\item\textbf{1974/1.} \\
Three players $A$, $B$ and $C$ play the following game: On each of three cards an integer is written. These three numbers $p, q, r$ satisfy $0 < p < q < r$. The three cards are shuffled and one is dealt to each player. Each then receives the number of counters indicated by the card he holds. Then the cards are shuffled again; the counters remain with the players. \\
This process (shuffling, dealing, giving out counters) takes place for at least two rounds. After the last round, $A$ has 20 counters in all, $B$ has 10 and $C$ has 9. At the last round $B$ received $r$ counters. Who received $q$ counters on the first round?

\bigskip

\item\textbf{1974/2.} \\
In the triangle $ABC$, prove that there is a point $D$ on side $AB$ such that $CD$ is the geometric mean of $AD$ and $DB$ if and only if
$
\sin A \sin B \leq \sin^2 \frac{C}{2}.
$

\bigskip

\item\textbf{1974/3.} \\
Prove that the number
$
\sum_{k=0}^{n} \binom{2n+1}{2k+1} \cdot 2^{3k}
$
is not divisible by 5 for any integer $n \geq 0$.

\bigskip

\item\textbf{1974/4.} \\
Consider decompositions of an $8 \times 8$ chessboard into $p$ non-overlapping rectangles subject to the following conditions:
\begin{itemize}
  \item[(i)] Each rectangle has as many white squares as black squares.
  \item[(ii)] If $a_i$ is the number of different rectangles in the 1\textsuperscript{st} rectangle, then $a_1 < a_2 < \dots < a_p$. 
\end{itemize}
Find the maximum value of $p$ for which such a decomposition is possible. For this value of $p$, determine all possible sequences $a_1, a_2, \dots, a_p$.

\bigskip

\item\textbf{1974/5.} \\
Determine all possible values of
$
S = \frac{a}{a+b+d} + \frac{b}{a+b+c} + \frac{c}{b+c+d} + \frac{d}{a+c+d}
$
where $a, b, c, d$ are arbitrary positive numbers.

\bigskip

\item\textbf{1974/6.} \\
Let $P$ be a non-constant polynomial with integer coefficients. If $n(P)$ is the number of distinct integers $k$ such that $(P(k))^2 = 1$, prove that
$
n(P) - \deg(P) \leq 2,
$
where $\deg(P)$ denotes the degree of the polynomial $P$.
\end{enumerate}

\section*{Seventeenth International Olympiad, 1975}
\begin{enumerate}
\item\textbf{1975/1.} \\
Let $x_i, y_i$ ($i = 1, 2, \ldots, n$) be real numbers such that
$
x_1 \ge x_2 \ge \cdots \ge x_n \quad \text{and} \quad y_1 \ge y_2 \ge \cdots \ge y_n.
$
Prove that, if $z_1, z_2, \ldots, z_n$ is any permutation of $y_1, y_2, \ldots, y_n$, then
$
\sum_{i=1}^{n} (x_i - y_i)^2 \le \sum_{i=1}^{n} (x_i - z_i)^2.
$

\bigskip

\item\textbf{1975/2.} \\
Let $a_1, a_2, a_3, \ldots$ be an infinite increasing sequence of positive integers. Prove that for every $p \ge 1$ there are infinitely many $a_m$ which can be written in the form
$
a_m = x a_p + y a_q
$
with $x, y$ positive integers and $q > p$.

\bigskip

\item\textbf{1975/3.} \\
On the sides of an arbitrary triangle $ABC$, triangles $ABR$, $BCP$, $CAQ$ are constructed externally with $\angle CBP = \angle CAQ = 45^\circ$, $\angle BCP = \angle ACQ = 30^\circ$, $\angle ABR = \angle BAR = 15^\circ$. Prove that $\angle QRP = 90^\circ$ and $QR = RP$.

\bigskip

\item\textbf{1975/4.} \\
When $4444^{4444}$ is written in decimal notation, the sum of its digits is $A$. Let $B$ be the sum of the digits of $A$. Find the sum of the digits of $B$. ($A$ and $B$ are written in decimal notation.)

\bigskip

\item\textbf{1975/5.} \\
Determine, with proof, whether or not one can find 1975 points on the circumference of a circle of unit radius such that the distance between any two of them is a rational number.

\bigskip

\item\textbf{1975/6.} \\
Find all polynomials $P$, in two variables, with the following properties: \\
(i) for a positive integer $n$ and all real $t, x, y$,
$
P(tx, ty) = t^n P(x, y).
$
\end{enumerate}
\section*{Eighteenth International Olympiad, 1976}
\begin{enumerate}
\item\textbf{1976/1.} \\
In a plane convex quadrilateral of area 32, the sum of the lengths of two opposite sides and one diagonal is 16. Determine all possible lengths of the other diagonal.

\bigskip

\item\textbf{1976/2.} \\
Let $P_1(x) = x^2 - 2$ and $P_j(x) = P_1(P_{j-1}(x))$ for $j = 2, 3, \ldots$. Show that, for any positive integer $n$, the roots of the equation $P_n(x) = x$ are real and distinct.

\bigskip

\item\textbf{1976/3.} \\
A rectangular box can be filled completely with unit cubes. If one places as many cubes as possible, each with volume 2, in the box, so that their edges are parallel to the edges of the box, one can fill exactly 40\% of the box. Determine the possible dimensions of all such boxes.

\bigskip

\item\textbf{1976/4.} \\
Determine, with proof, the largest number which is the product of positive integers whose sum is 1976.

\bigskip

\item\textbf{1976/5.} \\
Consider the system of $p$ equations in $q = 2p$ unknowns $x_1, x_2, \ldots, x_q$:
\[
\begin{aligned}
a_{11}x_1 + a_{12}x_2 + \cdots + a_{1q}x_q &= 0 \\
a_{21}x_1 + a_{22}x_2 + \cdots + a_{2q}x_q &= 0 \\
&\vdots \\
a_{p1}x_1 + a_{p2}x_2 + \cdots + a_{pq}x_q &= 0
\end{aligned}
\]
with every coefficient $a_{ij}$ member of the set $\{-1, 0, 1\}$. Prove that the system has a solution $(x_1, x_2, \ldots, x_q)$, not all zero, such that:
\begin{itemize}
  \item[(a)] all $x_j$ ($j=1, 2, \ldots, q$) are integers,
  \item[(b)] there is at least one value of $j$ for which $x_j \ne 0$,
  \item[(c)] $|x_j| \le q$ ($j = 1, 2, \ldots, q$).
\end{itemize}

\bigskip

\item\textbf{1976/6.} \\
A sequence $\{u_n\}$ is defined by
$
u_0 = 2, \quad u_1 = \frac{5}{2}, \quad u_{n+1} = u_n(u_{n-1}^2 - 2) - u_{1} \quad \text{for } n = 1, 2, \ldots
$
\end{enumerate}
\end{document}
